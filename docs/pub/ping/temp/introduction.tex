\section{Introduction}

The integration of software applications into the statistical production chain is a key question in a modern environment where standards and software may swiftly be outdated, thereby necessitating a thorough reflection in terms of transparency and documentation of the processes. We would like to exhibit and promote in this paper the implementation of a high-level collaborative platform aiming not only at producing social statistics, but also at further fostering experimentation and analysis in that field. In doing so, we strongly support the (obvious) claim of \cite{salgado2016modern} that "the modernisation and industrialisation of official statistical production needs a unified combination of statistics and computer science in its very principles".
\\
Motivated by the consensus that processes - in particular statistical processes \cite{sutherland2013twenty} - for data-driven policy should be transparent \cite{bertot2014big}, we require software development and deployment in a statistical organisation to be open, reusable verifiable, reproducible, and collaborative. Beyond just devising guidelines and best practices, we show how the platform is implemented for the production of social statistics. For that purpose, we adopt a reasonable mix of bottom-up (from low-level scope to high-level vision) and top-down (from black-box process models to traceable functional modules) designs, so as to "think global, [and] act local" \cite{lalor2013modernising}. In building the parts while planning the whole, we provide with a flexible and agile approach to immediate needs and current legacy issues, as well as long-term problems and potential future requirements in statistical production \cite{UNECE2015common}.
